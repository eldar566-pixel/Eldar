% ═══════════════════════════════════════════════════════════════════════════
%                         МАТЕМАТИЧЕСКИЙ АТЛАС
%                    Трактатный подход «АнтиБурбаки»
% ═══════════════════════════════════════════════════════════════════════════

\documentclass[12pt,a4paper,openany]{book}

% ═══════════════════════════════════════════════════════════════════════════
%                              ПАКЕТЫ
% ═══════════════════════════════════════════════════════════════════════════

% Кодировка и шрифты (XeLaTeX)
\usepackage{fontspec}
\setmainfont{DejaVu Serif}
\setsansfont{DejaVu Sans}
\setmonofont{DejaVu Sans Mono}

% Математика
\usepackage{amsmath,amssymb,amsthm}
\usepackage{mathtools}
\usepackage{mathrsfs}

% Графика и диаграммы
\usepackage{tikz}
\usepackage{tikz-cd}
\usetikzlibrary{shapes,arrows,positioning,calc,decorations.pathreplacing,matrix}
\usepackage{graphicx}

% Таблицы
\usepackage{booktabs}
\usepackage{array}
\usepackage{longtable}
\usepackage{multirow}
\usepackage{tabularx}

% Оформление
\usepackage{tcolorbox}
\tcbuselibrary{theorems,skins,breakable}
\usepackage{enumitem}
\usepackage{fancyhdr}
\usepackage{titlesec}

% Геометрия страницы
\usepackage[margin=2.5cm]{geometry}

% Гиперссылки
\usepackage{hyperref}
\hypersetup{
    colorlinks=true,
    linkcolor=blue!70!black,
    urlcolor=blue!70!black,
    citecolor=green!50!black
}

% Дополнительные символы
\usepackage{textcomp}

% ═══════════════════════════════════════════════════════════════════════════
%                        КОМАНДЫ И ОКРУЖЕНИЯ
% ═══════════════════════════════════════════════════════════════════════════

% Числовые множества
\newcommand{\N}{\mathbb{N}}
\newcommand{\Z}{\mathbb{Z}}
\newcommand{\Q}{\mathbb{Q}}
\newcommand{\R}{\mathbb{R}}
\newcommand{\C}{\mathbb{C}}
\newcommand{\HH}{\mathbb{H}}
\newcommand{\OO}{\mathbb{O}}

% Операторы
\DeclareMathOperator{\Ker}{Ker}
\DeclareMathOperator{\Ima}{Im}
\DeclareMathOperator{\id}{id}
\DeclareMathOperator{\Hom}{Hom}
\DeclareMathOperator{\End}{End}
\DeclareMathOperator{\Aut}{Aut}
\DeclareMathOperator{\GL}{GL}
\DeclareMathOperator{\SL}{SL}
\DeclareMathOperator{\SO}{SO}
\DeclareMathOperator{\SU}{SU}
\DeclareMathOperator{\rank}{rank}
\DeclareMathOperator{\tr}{tr}
\DeclareMathOperator{\sgn}{sgn}
\DeclareMathOperator{\grad}{grad}
\DeclareMathOperator{\dive}{div}
\DeclareMathOperator{\rot}{rot}

% Скалярное произведение и норма
\newcommand{\inner}[2]{\langle #1, #2 \rangle}
\newcommand{\norm}[1]{\| #1 \|}

% ═══════════════════════════════════════════════════════════════════════════
%                       СТИЛИ БОКСОВ (tcolorbox)
% ═══════════════════════════════════════════════════════════════════════════

\newtcolorbox{infobox}[1][]{
    enhanced, breakable,
    colback=blue!5, colframe=blue!70!black,
    fonttitle=\bfseries, title=#1,
    boxrule=1pt, arc=3pt, left=8pt, right=8pt, top=6pt, bottom=6pt
}

\newtcolorbox{defbox}[1][Определение]{
    enhanced, breakable,
    colback=green!5, colframe=green!50!black,
    fonttitle=\bfseries, title=#1,
    boxrule=1pt, arc=3pt, left=8pt, right=8pt
}

\newtcolorbox{thmbox}[1][Теорема]{
    enhanced, breakable,
    colback=red!5, colframe=red!50!black,
    fonttitle=\bfseries, title=#1,
    boxrule=1pt, arc=3pt, left=8pt, right=8pt
}

\newtcolorbox{exbox}[1][Пример]{
    enhanced, breakable,
    colback=yellow!10, colframe=orange!70!black,
    fonttitle=\bfseries, title=#1,
    boxrule=1pt, arc=3pt, left=8pt, right=8pt
}

\newtcolorbox{warnbox}[1][Внимание]{
    enhanced, breakable,
    colback=red!10, colframe=red!70!black,
    fonttitle=\bfseries, title={$\triangle$ #1},
    boxrule=1.5pt, arc=3pt, left=8pt, right=8pt
}

\newtcolorbox{linkbox}[1][Связь с другими разделами]{
    enhanced, breakable,
    colback=purple!5, colframe=purple!50!black,
    fonttitle=\bfseries, title=#1,
    boxrule=1pt, arc=3pt, left=8pt, right=8pt
}

\newtcolorbox{simplebox}{
    enhanced, breakable,
    colback=gray!5, colframe=gray!50!black,
    boxrule=0.5pt, arc=2pt, left=8pt, right=8pt, top=6pt, bottom=6pt
}

% ═══════════════════════════════════════════════════════════════════════════
%                         НАСТРОЙКИ ЗАГОЛОВКОВ
% ═══════════════════════════════════════════════════════════════════════════

\titleformat{\part}[display]
    {\centering\Huge\bfseries}{\partname\ \thepart}{20pt}{\Huge}

\titleformat{\chapter}[display]
    {\Large\bfseries}{\chaptertitlename\ \thechapter}{10pt}{\LARGE}

% ═══════════════════════════════════════════════════════════════════════════
%                         НАЧАЛО ДОКУМЕНТА
% ═══════════════════════════════════════════════════════════════════════════

\begin{document}

% ═══════════════════════════════════════════════════════════════════════════
%                         ТИТУЛЬНАЯ СТРАНИЦА
% ═══════════════════════════════════════════════════════════════════════════

\begin{titlepage}
\centering
\vspace*{1cm}

{\Huge\bfseries МАТЕМАТИЧЕСКИЙ АТЛАС}

\vspace{0.5cm}

{\Large\itshape Трактатный подход <<АнтиБурбаки>>}

\vspace{1.5cm}

\includegraphics[width=0.75\textwidth]{cover_image_real.png}

\vspace{1.5cm}

{\large Это карта математической территории.\\
Атлас показывает \textbf{связи} между разделами.}

\vfill

\end{titlepage}

\tableofcontents

% ═══════════════════════════════════════════════════════════════════════════
%                          ВВЕДЕНИЕ
% ═══════════════════════════════════════════════════════════════════════════

\chapter*{Введение: Что такое этот атлас}
\addcontentsline{toc}{chapter}{Введение}

\begin{infobox}[Математика — это язык пространств]
Комната, в которой мы сидим — это пространство.
Множество всех возможных температур в этой комнате — тоже пространство.
Множество всех функций, описывающих температуру — тоже пространство.

Математика изучает пространства разных типов и связи между ними.

Каждый раздел математики — это \textbf{способ смотреть} на пространство:

\begin{center}
\begin{tabular}{lp{9cm}}
\toprule
\textbf{Раздел} & \textbf{Что видит в пространстве} \\
\midrule
Теория множеств & Только точки, никакой структуры \\
Топология & Какие точки <<близки>> (но без чисел-расстояний) \\
Метрика & Расстояния между точками \\
Линейная алгебра & Сложение и умножение на числа \\
Группы & Симметрии — преобразования, сохраняющие структуру \\
Многообразия & Локально похоже на $\R^n$, глобально искривлено \\
Функц. анализ & Бесконечномерные пространства функций \\
\bottomrule
\end{tabular}
\end{center}

Одно и то же физическое пространство можно изучать всеми способами.
Разные задачи требуют разных взглядов.
\end{infobox}

Это карта математической территории. Атлас показывает \textbf{связи} между разделами.

\begin{infobox}[Нотации (чтобы не листать в конец)]
\begin{tabular}{llll}
$\in$ $\notin$ $\subset$ $\subseteq$ & принадлежит, включение & $\forall$ $\exists$ $\Rightarrow$ $\Leftrightarrow$ & кванторы, следование \\
$\cap$ $\cup$ $\setminus$ & пересечение, объединение & $\neg$ $\land$ $\lor$ & не, и, или \\
$\N$ $\Z$ $\Q$ $\R$ $\C$ & числовые множества & $\HH$ $\OO$ & кватернионы, октонионы \\
$\cong$ & изоморфизм & $\otimes$ $\oplus$ & тензорное, прямая сумма \\
$\inner{\cdot}{\cdot}$ $\norm{\cdot}$ & скалярное произв., норма & $V^*$ $A^\top$ $A^{-1}$ & двойств., трансп. \\
\end{tabular}
\end{infobox}

\begin{infobox}[Иерархия пространств — главная схема]
\begin{center}
\begin{tikzpicture}[node distance=1.2cm, every node/.style={align=center}]
\node (set) {\textbf{Множество}\\(точки без структуры)};
\node (top) [below left=1cm and 0.5cm of set] {Топологическое\\(близость)};
\node (alg) [below=1cm of set] {Алгебраическое\\(группа, кольцо)};
\node (ord) [below right=1cm and 0.5cm of set] {Порядок\\(решётка)};
\node (met) [below=1cm of top] {Метрическое\\(расстояние)};
\node (vec) [below=1cm of alg] {Векторное\\пространство};
\node (norm) [below=1cm of met, xshift=1.5cm] {Нормированное\\(длина вектора)};
\node (hilb) [below=1cm of norm] {Гильбертово\\(скалярное произведение)};
\node (man) [below=1cm of hilb] {Многообразие\\(локально евклидово)};
\node (riem) [below=1cm of man] {Риманово многообразие\\(метрика меняется)};

\draw[->] (set) -- (top);
\draw[->] (set) -- (alg);
\draw[->] (set) -- (ord);
\draw[->] (top) -- (met);
\draw[->] (alg) -- (vec);
\draw[->] (met) -- (norm);
\draw[->] (vec) -- (norm);
\draw[->] (norm) -- (hilb);
\draw[->] (hilb) -- (man);
\draw[->] (man) -- (riem);
\end{tikzpicture}
\end{center}
\end{infobox}

\begin{infobox}[Почему матанализ не в начале? (важное пояснение)]
Традиционное образование: школа $\to$ матанализ $\to$ всё остальное.
Этот атлас устроен \textbf{иначе}.

Матанализ — это анализ на \textbf{конкретном} пространстве $\R^n$.
Мы сначала отвечаем: что такое пространство \textbf{вообще}?

\begin{itemize}[nosep]
\item Топология: что значит <<близко>> и <<непрерывно>>
\item Линейная алгебра: что значит <<сложить>> и <<умножить на число>>
\item Группы: что значит <<симметрия>>
\item Многообразия: что значит <<локально как $\R^n$>>
\end{itemize}

И \textbf{потом} показываем: вот как всё это работает на $\R^n$ (матанализ).

Это как учить географию: можно начать с родного города, а можно — с понятий <<планета>>, <<континент>>, <<страна>>. Мы выбираем второй путь.

\textbf{Если ты хочешь начать с матанализа} — иди сразу в раздел A13.
Там есть всё: пределы, производные, интегралы, ряды.
Но потом вернись к A2 (топология) — увидишь, откуда это всё растёт.
\end{infobox}

\begin{infobox}[Одна задача — много языков (зачем нужен атлас)]
Теплопроводность в стержне. \textbf{Одна} физика, но:

\begin{center}
\begin{tabular}{lp{8cm}}
\toprule
\textbf{Язык} & \textbf{Как выглядит} \\
\midrule
Физика & Тепло течёт от горячего к холодному \\
Матанализ (A13) & $\partial T/\partial t = \alpha \cdot \partial^2 T/\partial x^2$ (уравнение в частных произв.) \\
Фурье (A14) & $T(x,t) = \sum c_n e^{-\alpha n^2 t} \sin(n\pi x/L)$ \\
Функанализ (A15) & $dT/dt = AT$, где $A = \alpha \cdot d^2/dx^2$ — оператор в $L^2$ \\
Полугруппы & $T(t) = e^{At}T_0$ — однопараметрическая полугруппа \\
Вероятность (A17) & Броуновское движение, диффузия частиц \\
\bottomrule
\end{tabular}
\end{center}

Все эти языки описывают \textbf{одно и то же}. Атлас показывает, как переходить между ними. Иногда задача проще на одном языке, иногда — на другом.
\end{infobox}

\begin{infobox}[Вторая задача — ещё больше языков (течение в трубе)]
Вода течёт по трубе. Какой расход? Какие потери давления?

\begin{center}
\begin{tabular}{lp{8cm}}
\toprule
\textbf{Раздел атласа} & \textbf{Что даёт для этой задачи} \\
\midrule
Векторы (A3) & Скорость $\vec{v} = (v_x, v_y, v_z)$ — векторное поле \\
Тензоры (A6) & Напряжения $\tau_{ij}$ — тензор 2-го ранга. Связь $\tau$ и скорости: $\tau_{ij} = \mu(\partial v_i/\partial x_j + \partial v_j/\partial x_i)$ \\
Формы (A8) & Расход $= \iint_S \vec{v} \cdot d\vec{S}$ — интеграл 2-формы $\rho v$ \\
Матанализ (A13) & Уравнение Навье-Стокса: $\rho(\partial\vec{v}/\partial t + (\vec{v}\cdot\nabla)\vec{v}) = -\nabla p + \mu\nabla^2\vec{v} + \rho\vec{g}$ \\
Функанализ (A15) & Слабые решения, пространства Соболева $W^{1,2}$ \\
Группы (A1) & Симметрия задачи: осевая $\to$ профиль Пуазейля $v(r) = v_{\max}(1 - r^2/R^2)$ \\
Размерности (A6) & Число Рейнольдса $Re = \rho v L/\mu$ — безразмерное! $Re < 2300$: ламинар, $Re > 4000$: турбулентность \\
\bottomrule
\end{tabular}
\end{center}

Инженерная формула Дарси-Вейсбаха: $\Delta P = \lambda \cdot (L/D) \cdot (\rho v^2/2)$

Откуда $\lambda$? Из решения Навье-Стокса или эмпирических корреляций.
\end{infobox}

% ═══════════════════════════════════════════════════════════════════════════
%                          ЧАСТЬ I: ФУНДАМЕНТ
% ═══════════════════════════════════════════════════════════════════════════

\part{Фундамент}

% ═══════════════════════════════════════════════════════════════════════════
%                     1. ФИЛОСОФСКИЙ ФУНДАМЕНТ
% ═══════════════════════════════════════════════════════════════════════════

\chapter{Философский фундамент}

\begin{infobox}[Иерархия мышления: от пустоты к физике]
Вся математика и познание мира возникают из последовательности актов мышления:

\begin{center}
\begin{tikzpicture}[node distance=1.5cm, every node/.style={align=center}]
\node (void) {$\varnothing$ \textbf{Пустота}};
\node (act1) [below=0.8cm of void] {\fbox{\parbox{8cm}{\centering Субъект наделяется возможностью действовать над пустотой}}};
\node (act2) [below=0.8cm of act1] {\fbox{\parbox{8cm}{\centering Пустота разрезается границами на \textbf{образы}}}};
\node (act3) [below=0.8cm of act2] {\fbox{\parbox{8cm}{\centering Универсальны для всего живого\\Не требуют символов\\Прямое оперирование паттернами}}};
\node (act4) [below=0.8cm of act3] {\fbox{\parbox{8cm}{\centering \textbf{Теория множеств}\\Минимальный язык для описания наборов объектов\\Мост между образами и коммуникацией}}};
\node (act5) [below=0.8cm of act4] {\fbox{\parbox{8cm}{\centering \textbf{Естественные языки}\\Символическое представление образов\\Потеря точности при передаче}}};
\node (act6) [below=0.8cm of act5] {\fbox{\parbox{8cm}{\centering \textbf{Логика / Доказательства}\\Нужны из-за ненадёжности языка\\Попытка восстановить исходную ясность образов}}};
\node (act7) [below=0.8cm of act6] {\fbox{\parbox{8cm}{\centering \textbf{Физика}\\Экспериментальная наука с высокой воспроизводимостью\\Математика = экспериментальная физика}}};

\draw[->] (void) -- node[right] {Акт 1: Выбор} (act1);
\draw[->] (act1) -- node[right] {Акт 2: Проведение границ} (act2);
\draw[->] (act2) -- node[right] {Акт 3: Манипуляции образами} (act3);
\draw[->] (act3) -- node[right] {Акт 4: Категоризация} (act4);
\draw[->] (act4) -- node[right] {Акт 5: Коммуникация} (act5);
\draw[->] (act5) -- node[right] {Акт 6: Проверка коммуникации} (act6);
\draw[->] (act6) -- node[right] {Акт 7: Применение к миру} (act7);
\end{tikzpicture}
\end{center}
\end{infobox}

\begin{center}
\textbf{Ключевые философские положения}

\begin{tabular}{cp{12cm}}
\toprule
\textbf{№} & \textbf{Положение} \\
\midrule
1 & Вселенная есть пустота, бесконечно переопределяющая границы \\
2 & Существование объекта = возможность кому-то указать на этот объект \\
3 & Мышление = указание, к каким множествам относятся объекты \\
4 & Доказательство = явный путь по карте вложений множеств \\
5 & Логика и математика = экспериментальная физика с высокой воспроизводимостью \\
6 & Понять = уметь визуально представить \\
7 & <<Объект>> и <<пространство>> — не свойства вещи, а роли, определяемые актом проведения границы наблюдателем (см. раздел ниже) \\
\bottomrule
\end{tabular}
\end{center}

\begin{warnbox}[Примечание о философии математики]
Положение 2 (<<Существование = возможность указать>>) ближе к \textbf{конструктивизму} (Брауэр, Гейтинг), чем к классической математике.

В \textbf{классической математике} (ZFC) <<существование>> ($\exists$) слабее:
\begin{itemize}[nosep]
\item Достаточно доказать, что несуществование приводит к противоречию
\item Не обязательно уметь <<показать>> объект явно
\end{itemize}

\textbf{Пример:} Доказательство существования иррационального $a^b$ ($\sqrt{2}^{\sqrt{2}}$):

Либо $\sqrt{2}^{\sqrt{2}}$ рационально, либо $(\sqrt{2}^{\sqrt{2}})^{\sqrt{2}} = 2$ рационально.
Значит, такая пара $(a,b)$ \textbf{существует}, хотя мы не знаем какая!
Это классическое, но \textbf{не}конструктивное доказательство.

В этом атласе используется классическая математика (ZFC), но философский взгляд остаётся интуитивно конструктивным — мы стремимся не просто доказать существование, а \textbf{показать} объект.
\end{warnbox}

\section{Объект или пространство? — Вопрос точки зрения}

Философское положение 1 гласит: <<Вселенная есть пустота, бесконечно переопределяющая границы>>

Отсюда следует глубокий вывод: \textbf{одна и та же сущность} может быть и объектом, и пространством — в зависимости от того, \textbf{где} мы проводим границу наблюдения.

\subsection{Пример: бублик (тор)}

\textbf{Бублик как объект:}
\begin{itemize}[nosep]
\item Мы смотрим на него <<снаружи>>
\item Он — элемент пространства всех поверхностей
\item Нас интересует он \textbf{целиком}: его род, площадь, как вложен в $\R^3$
\item Граница проведена \textbf{вокруг} бублика
\end{itemize}

\begin{center}
\begin{tikzpicture}
\draw (0,0) rectangle (6,2);
\node at (1,1) {$\bullet$};
\node at (1,0.5) {\small сфера};
\node at (3,1) {$\bullet$};
\node at (3,0.5) {\small тор};
\node at (5,1) {$\bullet$};
\node at (5,0.5) {\small двойной тор};
\node at (3,2.3) {Пространство поверхностей};
\node at (7,1) {$\leftarrow$ каждая поверхность = точка};
\end{tikzpicture}
\end{center}

\textbf{Бублик как пространство:}
\begin{itemize}[nosep]
\item Мы <<живём>> на нём
\item Нас интересуют \textbf{точки} на нём, пути между ними, функции на нём
\item Муравей ползает по бублику — для муравья это пространство
\item Граница проведена \textbf{внутри} бублика (между его точками)
\end{itemize}

\begin{center}
\begin{tikzpicture}
\draw (0,0) ellipse (2 and 1);
\node at (-0.5,0.3) {$\bullet p$};
\node at (0.5,0.3) {$\bullet q$};
\draw[->, dashed] (-0.5,0.3) to[bend left] (0.5,0.3);
\node at (0,-0.2) {\small путь};
\node at (0,1.5) {Поверхность тора};
\node at (4,0) {$\leftarrow$ каждая точка = объект};
\end{tikzpicture}
\end{center}

\textbf{Оба взгляда верны!} Бублик один, но акт выбора границы определяет роль.

\subsection{Рекурсия: пространство становится объектом}

\begin{center}
\begin{tabular}{ll}
Уровень 0: & Точка $p$ на торе $T$ — $p$ \textbf{объект} \\
Уровень 1: & Тор $T$ — $T$ \textbf{пространство} для $p$ \\
Уровень 2: & Пространство модулей торов — $T$ становится \textbf{объектом}! \\
Уровень 3: & Пространство всех модулей — предыдущее становится объектом \\
$\vdots$ & \\
\end{tabular}
\end{center}

На каждом уровне то, что было пространством, становится объектом в пространстве более высокого уровня. Граница поднимается.

\subsection{Аналогия с городом}

\begin{center}
\begin{tabular}{lp{9cm}}
\toprule
\textbf{Точка зрения} & \textbf{Город — это...} \\
\midrule
Пилот самолёта & \textbf{Объект} (точка на карте среди других городов) \\
Пешеход внутри & \textbf{Пространство} (улицы, маршруты, точки назначения) \\
Архитектор & \textbf{Объект} в пространстве градостроительных проектов \\
\bottomrule
\end{tabular}
\end{center}

Один и тот же город! Но пилот оперирует множеством городов, а пешеход — множеством точек внутри города.

\subsection{Пример из физики: жидкость и газ}

Этот пример особенно важен, потому что показывает, как \textbf{один и тот же} физический объект требует \textbf{разных} математических описаний.

\begin{center}
\begin{tabular}{lp{9cm}}
\toprule
\textbf{Подход} & \textbf{Жидкость/газ — это...} \\
\midrule
Термодинамика & \textbf{Объект} с параметрами $(P, V, T, S)$. <<Чему равно давление газа в баллоне?>> Внутренняя структура неважна — только состояние. \\
\midrule
Гидродинамика (уравнения Навье-Стокса) & \textbf{Пространство} с полями $v(x,t)$, $P(x,t)$, $\rho(x,t)$. <<Как течёт жидкость вокруг крыла?>> Каждая точка — место, где определены скорость, давление, плотность. \\
\midrule
Кинетическая теория & \textbf{Пространство молекул} (фазовое пространство). Каждая молекула — объект с координатами $(x, v)$. Газ = облако точек в $6N$-мерном пространстве! \\
\bottomrule
\end{tabular}
\end{center}

\textbf{Ключевое наблюдение:}
\begin{itemize}[nosep]
\item В термодинамике: газ = точка в пространстве состояний $(P,V,T)$
\item В гидродинамике: газ = само пространство, где живут поля
\item В кинетике: газ = множество частиц, каждая из которых объект
\end{itemize}

Три \textbf{разных} уровня описания — три разных ответа на вопрос <<что здесь объект, а что пространство>>.

Уравнения Навье-Стокса, уравнение Больцмана, уравнение состояния — это \textbf{не} конкуренты, а описания на \textbf{разных уровнях} иерархии!

\subsection{Практический критерий}

\textbf{Объект} — когда спрашиваем <<какой он?>> (свойства целого)

\textbf{Пространство} — когда спрашиваем <<что в нём?>> (структура внутри)

Это не свойство вещи, а свойство \textbf{вопроса}, который мы задаём.

\subsection{Связь с сечениями Дедекинда (см. раздел 5.1)}

Там показано: <<Иррациональное число — это не объект, а \textbf{граница} между объектами. $\sqrt{2}$ — это не конкретная дробь, а место разреза числовой прямой.>>

Это та же идея! Граница (акт разрезания) создаёт новую сущность. И эта сущность может быть объектом или пространством — в зависимости от того, на каком уровне мы работаем.

\subsection{Категорный взгляд (см. A12)}

Теория категорий делает эту двойственность явной и формальной:
\begin{itemize}[nosep]
\item В категории Top тор $T$ — \textbf{объект}
\item Пространство функций $C(T, \R)$ — новое пространство, где \textbf{функции} становятся объектами
\item Функционалы на $C(T, \R)$ — ещё более высокий уровень
\end{itemize}

Категория = формализация того, <<кто объект, а кто пространство>> на данном уровне абстракции.

\subsection{Философский вывод}

<<Объект>> и <<пространство>> — не абсолютные свойства, а \textbf{роли}. Роль определяется актом проведения границы наблюдателем.

Это согласуется с положением 2: <<Существование объекта = возможность кому-то \textbf{указать} на этот объект.>> Чтобы указать, нужно выделить. Чтобы выделить, нужно провести границу. Граница определяет уровень.

Математика изучает структуры на \textbf{всех} уровнях одновременно — и даёт язык для перехода между ними.

% ═══════════════════════════════════════════════════════════════════════════
%                 1.1 ТЕОРИЯ МНОЖЕСТВ — БАЗОВЫЕ ПОНЯТИЯ
% ═══════════════════════════════════════════════════════════════════════════

\section{Теория множеств — базовые понятия}

\begin{defbox}[Что такое множество]
\textbf{Множество} — это совокупность объектов, рассматриваемых как единое целое.

Объекты, входящие в множество, называются его \textbf{элементами}.

\textbf{Способы задания:}
\begin{itemize}[nosep]
\item Перечисление: $A = \{1, 2, 3\}$
\item Описание свойства: $B = \{x : x > 0\}$ = <<все положительные $x$>>
\end{itemize}

\textbf{Особые множества:}
\begin{itemize}[nosep]
\item $\varnothing = \{\}$ — \textbf{пустое множество} (не содержит элементов)
\item $U$ — \textbf{универсум} (множество всех рассматриваемых объектов)
\end{itemize}
\end{defbox}

\begin{infobox}[Два главных отношения]
\textbf{$x \in A$} — <<$x$ — элемент множества $A$>>

$x$ — это один объект, лежащий внутри $A$.

\begin{center}
\begin{tikzpicture}
\draw (0,0) rectangle (2,1.5);
\node at (1,1.2) {$A$};
\node at (0.8,0.5) {$\bullet$};
\node at (1.1,0.5) {$x$};
\node at (3,0.75) {$\leftarrow$ точка $x$ внутри $A$};
\end{tikzpicture}
\end{center}

\textbf{$B \subseteq A$} — <<$B$ — подмножество $A$>>

Каждый элемент $B$ является элементом $A$ ($B$ целиком лежит внутри $A$).

\begin{center}
\begin{tikzpicture}
\draw (0,0) rectangle (3,2);
\node at (1.5,1.7) {$A$};
\draw (0.7,0.4) rectangle (1.8,1.2);
\node at (1.25,0.8) {$B$};
\node at (4,1) {$\leftarrow$ $B$ целиком внутри $A$};
\end{tikzpicture}
\end{center}
\end{infobox}

\begin{warnbox}[Важно не путать]
\begin{itemize}[nosep]
\item $x \in A$ — $x$ это \textbf{объект} внутри $A$
\item $B \subseteq A$ — $B$ это \textbf{множество}, все элементы которого в $A$
\end{itemize}

\textbf{Пример:} $A = \{1, 2, 3\}$
\begin{itemize}[nosep]
\item $2 \in A$ — \textbf{да} ($2$ — элемент $A$)
\item $\{2\} \subseteq A$ — \textbf{да} ($\{2\}$ — подмножество $A$)
\item $\{2\} \in A$ — \textbf{нет!} ($\{2\}$ не является элементом $A$, элементы — числа)
\item $2 \subseteq A$ — \textbf{не имеет смысла} ($2$ — не множество)
\end{itemize}

\textbf{Пустое множество:}
\begin{itemize}[nosep]
\item $\varnothing \subseteq A$ — \textbf{всегда верно} для любого $A$
\item $\varnothing \in A$ — верно \textbf{только} если $\varnothing$ явно указано как элемент
\end{itemize}
\end{warnbox}

\begin{center}
\textbf{Операции над множествами}

\begin{tabular}{llp{4cm}}
\toprule
\textbf{Операция} & \textbf{Определение} & \textbf{Смысл} \\
\midrule
$A \cup B$ (объединение) & $\{x : x \in A \text{ или } x \in B\}$ & Все элементы, которые хотя бы в одном \\
$A \cap B$ (пересечение) & $\{x : x \in A \text{ и } x \in B\}$ & Все элементы, которые в обоих множествах \\
$A \setminus B$ (разность) & $\{x : x \in A \text{ и } x \notin B\}$ & Элементы $A$, которых нет в $B$ \\
$A^c$ (дополнение) & $\{x : x \notin A\}$ & Все элементы универсума, не входящие в $A$. $A^c = U \setminus A$ \\
\bottomrule
\end{tabular}
\end{center}

\begin{infobox}[Законы теории множеств]
\textbf{Коммутативность:} $A \cup B = B \cup A$, \quad $A \cap B = B \cap A$

\textbf{Ассоциативность:} $(A \cup B) \cup C = A \cup (B \cup C)$, \quad $(A \cap B) \cap C = A \cap (B \cap C)$

\textbf{Дистрибутивность:} $A \cap (B \cup C) = (A \cap B) \cup (A \cap C)$, \quad $A \cup (B \cap C) = (A \cup B) \cap (A \cup C)$

\textbf{Законы де Моргана:}
\begin{align*}
(A \cup B)^c &= A^c \cap B^c \quad \text{<<не($A$ или $B$)>> = <<не $A$ и не $B$>>} \\
(A \cap B)^c &= A^c \cup B^c \quad \text{<<не($A$ и $B$)>> = <<не $A$ или не $B$>>}
\end{align*}

\textbf{Свойства пустого и универсума:}
\begin{align*}
A \cup \varnothing &= A & A \cap \varnothing &= \varnothing \\
A \cup U &= U & A \cap U &= A \\
A \cup A^c &= U & A \cap A^c &= \varnothing
\end{align*}
\end{infobox}

\begin{infobox}[Мощность множества]
$|A|$ — количество элементов в конечном множестве $A$.

\textbf{Примеры:}
\begin{itemize}[nosep]
\item $|\varnothing| = 0$
\item $|\{a, b, c\}| = 3$
\item $|\{1, 2, \{1,2\}\}| = 3$ (три элемента: $1$, $2$, и множество $\{1,2\}$)
\end{itemize}

\textbf{Формула включений-исключений:} $|A \cup B| = |A| + |B| - |A \cap B|$

\textbf{Мощность множества подмножеств:} Если $|A| = n$, то $A$ имеет $2^n$ подмножеств (включая $\varnothing$ и само $A$).
\end{infobox}

\begin{infobox}[Декартово произведение]
$A \times B = \{(a, b) : a \in A, b \in B\}$

Множество всех упорядоченных пар, где первый элемент из $A$, второй из $B$.

\textbf{Пример:} $\{1, 2\} \times \{a, b\} = \{(1,a), (1,b), (2,a), (2,b)\}$

\textbf{Геометрически:} Если $A$ и $B$ — отрезки на осях, то $A \times B$ — прямоугольник.

$|A \times B| = |A| \cdot |B|$

$\R \times \R = \R^2$ — плоскость

$\R \times \R \times \R = \R^3$ — трёхмерное пространство
\end{infobox}

% ═══════════════════════════════════════════════════════════════════════════
%            1.2 АКСИОМАТИЧЕСКИЙ МЕТОД — ПРАВИЛА ИГРЫ В МАТЕМАТИКУ
% ═══════════════════════════════════════════════════════════════════════════

\section{Аксиоматический метод — правила игры в математику}

\begin{defbox}[Что такое аксиома]
\textbf{Аксиома} — это утверждение, которое принимается \textbf{без доказательства}.

Почему без доказательства? Потому что любое доказательство опирается на какие-то предшествующие утверждения. Если требовать доказывать всё, получим бесконечную цепочку или замкнутый круг. Нужна точка опоры.

\textbf{Аналогия: Правила настольной игры}
\begin{itemize}[nosep]
\item Вы не <<доказываете>>, что в шахматах конь ходит буквой Г
\item Это \textbf{правило игры} — мы его принимаем, чтобы играть
\item Аксиомы = правила математической игры
\item Теоремы = всё, что можно вывести из правил
\end{itemize}

\textbf{Важно:} Аксиомы не <<истинны>> и не <<ложны>> в абсолютном смысле. Они — соглашения. Разные наборы аксиом дают разные математики.
\end{defbox}

\begin{infobox}[Зачем нужен аксиоматический метод]
\textbf{История:} До конца XIX века математики работали <<интуитивно>>. Потом обнаружились парадоксы — логические противоречия.

\textbf{Парадокс Рассела (1901):}

Пусть $R = \{x : x \notin x\}$ — <<множество всех множеств, не содержащих себя>>

Вопрос: $R \in R$ или $R \notin R$?
\begin{itemize}[nosep]
\item Если $R \in R$, то по определению $R$ должно быть $R \notin R$. Противоречие!
\item Если $R \notin R$, то по определению $R$ должно быть $R \in R$. Противоречие!
\end{itemize}

\textbf{Вывод:} Нельзя просто так создавать <<множество всего, что удовлетворяет условию>>. Нужны \textbf{ограничения} — аксиомы, определяющие, какие множества <<легальны>>.

\textbf{Решение:} Система аксиом ZFC (Цермело-Френкель с аксиомой выбора)
\end{infobox}

\subsection{Аксиомы ZFC — фундамент современной математики}

ZFC = Zermelo-Fraenkel + Choice (Цермело-Френкель + Выбор)

Это система из $\sim$9 аксиом, на которой построена почти \textbf{вся} современная математика.

\begin{infobox}[Нотация (подробно — в разделе 2 <<Логика>>)]
\begin{tabular}{ll}
$\forall x$ & <<для всех $x$>> (квантор всеобщности) \\
$\exists x$ & <<существует $x$>> (квантор существования) \\
$\to$ & <<следует>>, <<влечёт>> \\
$\leftrightarrow$ & <<эквивалентно>>, <<тогда и только тогда>> \\
$\land$ & <<и>> (конъюнкция) \\
$\lor$ & <<или>> (дизъюнкция) \\
\end{tabular}
\end{infobox}

\begin{center}
\textbf{Аксиомы ZFC — сводная таблица}

\begin{tabular}{lp{4cm}p{5cm}}
\toprule
\textbf{Аксиома} & \textbf{Формула} & \textbf{Что даёт} \\
\midrule
1. Объёмность (Extensionality) & $(\forall x: x \in A \leftrightarrow x \in B) \to A = B$ & Множество = его элементы. $\{1,2,3\} = \{3,1,2\}$ \\
2. Пустое мн. (Empty Set) & $\exists \varnothing: \forall x: x \notin \varnothing$ & <<Ноль>> теории множеств. Точка отсчёта \\
3. Пара (Pairing) & $\forall a \forall b \, \exists P: x \in P \leftrightarrow (x=a \lor x=b)$ & Можно создать $\{a,b\}$ \\
4. Объединение (Union) & $\forall A \, \exists U: x \in U \leftrightarrow \exists B(B \in A \land x \in B)$ & Можно слить множества. $\bigcup\{\{1,2\},\{3\}\} = \{1,2,3\}$ \\
5. Булеан (Power Set) & $\forall A \, \exists P: B \in P \leftrightarrow B \subseteq A$ & Множество всех подмножеств. $|\mathcal{P}(A)| = 2^{|A|}$ \\
6. Бесконечн. (Infinity) & $\exists I: \varnothing \in I \land (x \in I \to x \cup \{x\} \in I)$ & Бесконечные множества сущ-ют. Мин. такое $I = \N$ \\
7. Выделение (Separation) & $\forall A \, \exists B: x \in B \leftrightarrow (x \in A \land \varphi(x))$ & $\{x \in A: \varphi(x)\}$ — легально. $\{x: \varphi(x)\}$ — нет! \\
8. Замена (Replacement) & Образ множества под функцией — множество & $\{f(a): a \in A\}$ — множество \\
9. Регулярн. (Foundation) & $A \neq \varnothing \to \exists x \in A: x \cap A = \varnothing$ & Нет $x \in x$, нет циклов. Всё строится из $\varnothing$ \\
\bottomrule
\end{tabular}
\end{center}

\begin{warnbox}[Важная тонкость: схемы аксиом]
Аксиомы 7 (Выделение) и 8 (Замена) — это \textbf{не} одиночные аксиомы, а \textbf{схемы}: для каждой формулы $\varphi(x)$ получается своя аксиома!
\begin{itemize}[nosep]
\item <<Выделение с $\varphi(x)$ = ($x$ чётное)>> — одна аксиома
\item <<Выделение с $\varphi(x)$ = ($x$ простое)>> — другая аксиома
\item ... и так далее для каждой возможной формулы
\end{itemize}

Поэтому ZFC формально содержит \textbf{бесконечно много} аксиом. Это не проблема — мы всё равно можем проверить любое конкретное доказательство за конечное время.
\end{warnbox}

\begin{warnbox}[Парадокс Рассела и как ZFC его решает]
\textbf{Парадокс:} Пусть $R = \{x : x \notin x\}$. Тогда $R \in R \Leftrightarrow R \notin R$. Противоречие!

\textbf{Решение в ZFC:}

Аксиома Выделения (Separation) запрещает строить $\{x : \varphi(x)\}$. Можно только $\{x \in A : \varphi(x)\}$ — подмножество \textbf{уже существующего} $A$.

Чтобы построить $R = \{x : x \notin x\}$, нужно сначала иметь множество $A$, содержащее $R$. Но такого $A$ не существует (по Регулярности $x \notin x$ всегда).

\textbf{Мораль:} ZFC — это <<свобода с ответственностью>>. Нельзя создавать множества <<из воздуха>> — только из уже построенных.
\end{warnbox}

\begin{center}
\textbf{Философский смысл аксиом}

\begin{tabular}{lp{10cm}}
\toprule
\textbf{Аксиома} & \textbf{Философская интерпретация} \\
\midrule
Пустое мн-во & Формализация <<пустоты>> — постулируем, что она существует \\
Бесконечность & Бесконечность — не очевидность, а \textbf{выбор} играть в такую математику. Можно строить математику без этой аксиомы \\
Выделение & Ограничение на <<наивное>> создание множеств — урок из парадокса Рассела. Свобода с ответственностью \\
Регулярность & Всё строится иерархически из $\varnothing$. Нет <<висящих в воздухе>> или самореферентных конструкций \\
\bottomrule
\end{tabular}
\end{center}

\begin{infobox}[Визуализация: кумулятивная иерархия — <<башня множеств>>]
Аксиомы ZFC порождают \textbf{всю} математику из \textbf{ничего} (пустого множества). Это происходит уровень за уровнем — кумулятивная иерархия $V_\alpha$.

\begin{center}
\begin{tabular}{l|p{10cm}}
$V_{\omega+1}$ & $\mathcal{P}(V_\omega)$ — степень бесконечного множества. Включает $\R$, все функции $\N \to \N$. \textbf{Это уже несчётно!} \\
\hline
$V_\omega$ & $V_0 \cup V_1 \cup V_2 \cup \ldots$ — первый бесконечный уровень. Включает $\N$, все конечные множества \\
\hline
$\vdots$ & $\vdots$ \\
\hline
$V_3$ & $\mathcal{P}(V_2)$ = все подмножества $V_2$. $|V_3| = 2^2 = 4$ элемента \\
\hline
$V_2$ & $\mathcal{P}(V_1) = \{\varnothing, \{\varnothing\}\}$. 2 элемента \\
\hline
$V_1$ & $\mathcal{P}(V_0) = \mathcal{P}(\varnothing) = \{\varnothing\}$. 1 элемент \\
\hline
$V_0$ & $\varnothing$ — пустое множество, начало всего. 0 элементов \\
\end{tabular}
\end{center}

\textbf{Рост размеров:}
$|V_0| = 0$, $|V_1| = 1$, $|V_2| = 2$, $|V_3| = 4$, $|V_4| = 16$, $|V_5| = 65536$, $|V_6| = 2^{65536} \approx 10^{19728}$ — уже непредставимо! $|V_\omega| = \aleph_0$, $|V_{\omega+1}| = 2^{\aleph_0} = |\R|$

\textbf{Мораль:} Вся математика <<вырастает>> из $\varnothing$ применением одной операции $\mathcal{P}$.
\end{infobox}

\begin{center}
\textbf{Где живут знакомые объекты}

\begin{tabular}{ll}
\toprule
\textbf{Объект} & \textbf{Первый уровень, где появляется} \\
\midrule
$\varnothing = 0$ & $V_0$ \\
$1 = \{\varnothing\}$ & $V_1$ \\
$2 = \{\varnothing, \{\varnothing\}\}$ & $V_2$ \\
$n$ (любое конечное) & $V_n$ \\
$\N$ (как множество) & $V_\omega$ \\
Функция $f: \N \to \N$ & $V_{\omega+1}$ (как подмножество $\N \times \N$) \\
$\R$ (как мн-во Дедекинда) & $V_{\omega+1}$ \\
Функция $f: \R \to \R$ & $V_{\omega+2}$ \\
Пространство $C[0,1]$ & $V_{\omega+2}$ \\
\bottomrule
\end{tabular}
\end{center}

Почти вся <<рабочая>> математика живёт в $V_{\omega+\omega}$ — относительно низко! Большие кардиналы (недостижимые, измеримые, ...) требуют \textbf{очень} высоких $V_\alpha$.

\subsection{Аксиома выбора — особый статус}

\begin{thmbox}[Аксиома выбора (AC)]
\textbf{Формула:} $\forall \mathcal{A}: (\forall A \in \mathcal{A}: A \neq \varnothing) \to \exists f: \forall A \in \mathcal{A}: f(A) \in A$

\textbf{Смысл:} Для любого семейства непустых множеств \textbf{существует} функция, выбирающая по одному элементу из каждого.

\textbf{Важно:} Слово <<одновременно>> — психологическое, а не математическое! Математически: существует \textbf{функция выбора} $f$. Никакого <<процесса выбирания>> нет — функция просто \textbf{есть} (или нет).

\textbf{Без AC} в некоторых моделях ZF: произведение непустых множеств \textbf{пусто}! Это не <<мы не можем выбрать>>, а <<выбора буквально не существует>>.
\end{thmbox}

\begin{center}
\textbf{Почему аксиома выбора неочевидна}

\begin{tabular}{lp{8cm}}
\toprule
\textbf{Случай} & \textbf{Ситуация} \\
\midrule
Конечное число коробок & Очевидно — просто перебираем (AC \textbf{не нужна!}) \\
Счётное число & Достаточно аксиомы счётного выбора ($\text{AC}_\omega$) — более слабой \\
Несчётное число & \textbf{Как} выбрать? Алгоритма нет! AC утверждает, что функция \textbf{существует}, даже без явного построения \\
\bottomrule
\end{tabular}
\end{center}

\begin{center}
\textbf{Эквиваленты аксиомы выбора}

\begin{tabular}{lp{8cm}}
\toprule
\textbf{Утверждение} & \textbf{Формулировка} \\
\midrule
Аксиома выбора & Можно одновременно выбрать по элементу из каждого множества семейства \\
Лемма Цорна & Если каждая цепь в ч.у. множестве имеет верхнюю грань, то есть максимальный элемент \\
Теорема о полном упорядочении & Любое множество можно вполне упорядочить (сравнить любые два элемента) \\
Произведение непустых множеств непусто & Если каждое $A_i \neq \varnothing$, то $\prod A_i \neq \varnothing$ \\
\bottomrule
\end{tabular}
\end{center}

\begin{infobox}[Что следует из аксиомы выбора]
\textbf{Хорошие следствия} (нужны в математике):
\begin{itemize}[nosep]
\item Любое векторное пространство имеет базис
\item Любой идеал содержится в максимальном идеале
\item Теорема Тихонова: произведение компактов компактно
\item Теорема Хана-Банаха: продолжение линейных функционалов
\item Любые два кардинала сравнимы: $|A| \leq |B|$ или $|B| \leq |A|$
\end{itemize}

\textbf{Парадоксальные следствия} (странные, но не противоречивые):
\begin{itemize}[nosep]
\item Парадокс Банаха-Тарского: шар можно разрезать на 5 частей и собрать \textbf{два} таких же шара! (части — не <<обычные>> куски, а патологические множества точек)
\item Существуют неизмеримые множества (по Лебегу, см. A16)
\end{itemize}

\textbf{Статус:} Аксиома выбора \textbf{независима} от ZF — её нельзя ни доказать, ни опровергнуть. Можно работать с ней (ZFC) или без неё (ZF). Большинство математиков принимают AC.
\end{infobox}

\begin{infobox}[<<Выбор>> в философии vs <<выбор>> в аксиоме]
В \textbf{философском фундаменте} (Акт 1) <<выбор>> = способность субъекта различать, проводить границы, выделять объекты из пустоты. Это \textbf{онтологический} выбор.

В \textbf{аксиоме выбора} <<выбор>> = существование функции, сопоставляющей элемент каждому множеству. Это \textbf{техническая} возможность.

\textbf{Связь:}
Философский выбор — первичен (без него нет математики вообще).
Аксиома выбора — конкретное правило игры, которое можно принять или нет.

\textbf{Разница:}
\begin{itemize}[nosep]
\item Философский выбор: <<Я \textbf{могу} выделить объект>>
\item Аксиома выбора: <<Для \textbf{любого} (даже несчётного) семейства \textbf{существует} способ одновременного выбора>>
\end{itemize}
\end{infobox}

\begin{infobox}[Резюме: зачем всё это нужно]
Аксиомы ZFC — это \textbf{не} <<истины о мире>>. Это \textbf{правила игры}, которые:
\begin{enumerate}[nosep]
\item Достаточно сильны, чтобы построить всю известную математику
\item Достаточно ограничены, чтобы избежать парадоксов
\item (Насколько известно) непротиворечивы
\item \textbf{Неполны} (теорема Гёделя, 1931): существуют истинные утверждения о натуральных числах, недоказуемые в ZFC
\end{enumerate}

Неполнота — не баг, а фундаментальное свойство любой достаточно богатой формальной системы. ZFC не может доказать свою непротиворечивость изнутри.

Когда далее мы пишем <<существует множество $X$>> или <<выберем элемент из $Y$>>, мы опираемся на эти аксиомы. Математика — это следствия этих правил.

Теперь, имея фундамент, можно строить числа.
\end{infobox}

\subsection{Теоремы Гёделя — границы формальных систем}

\begin{infobox}[Контекст: зачем это в атласе]
Теоремы Гёделя (1931) — это \textbf{не} абстрактная логика. Это фундаментальные результаты о том, \textbf{что может} и \textbf{чего не может} математика:
\begin{itemize}[nosep]
\item Почему некоторые задачи принципиально неразрешимы
\item Почему машина не может заменить математика полностью
\item Почему <<истина>> и <<доказуемость>> — разные вещи
\item Почему мы не можем быть уверены в непротиворечивости математики
\end{itemize}

Для инженера: если задача неразрешима — это не наша вина, а граница теории.
\end{infobox}

\begin{thmbox}[Первая теорема Гёделя о неполноте]
\textbf{Формулировка (упрощённая):}

В любой непротиворечивой формальной системе, достаточно богатой, чтобы описать арифметику натуральных чисел, существует утверждение $G$, которое:
\begin{itemize}[nosep]
\item \textbf{Истинно} (в стандартной интерпретации)
\item \textbf{Недоказуемо} в этой системе
\end{itemize}

\textbf{Интуиция — <<парадокс лжеца>>:}

Рассмотрим предложение: <<Это предложение ложно.>>
\begin{itemize}[nosep]
\item Если оно истинно — значит оно ложно (противоречие)
\item Если оно ложно — значит оно истинно (противоречие)
\end{itemize}

Гёдель построил \textbf{математический} аналог: <<Это утверждение недоказуемо в системе $S$.>>
\begin{itemize}[nosep]
\item Если оно доказуемо — мы доказали ложь (система противоречива)
\item Если система непротиворечива — оно недоказуемо
\item Но тогда оно \textbf{истинно} (ведь оно говорит о своей недоказуемости)!
\end{itemize}

\textbf{Следствие:} ZFC (и любая <<разумная>> система) \textbf{неполна}. Существуют истинные утверждения о числах, которые ZFC не докажет.
\end{thmbox}

\begin{thmbox}[Вторая теорема Гёделя о неполноте]
\textbf{Формулировка:}

Непротиворечивая система, достаточно богатая для арифметики, \textbf{не может доказать свою собственную непротиворечивость}.

\textbf{Интуиция:}

Утверждение $\text{Con}(S)$ = <<Система $S$ непротиворечива>> можно записать как арифметическую формулу.

Гёдель показал: $\text{Con}(S) \to G$

Если бы $S$ доказывала $\text{Con}(S)$, она доказала бы $G$. Но $G$ недоказуема (по первой теореме). Значит, $\text{Con}(S)$ тоже недоказуема в $S$.

\textbf{Философский смысл:}

Математика \textbf{не может} гарантировать свою собственную надёжность. Мы не можем \textbf{доказать}, что ZFC не приведёт к противоречию. Мы просто \textbf{верим} в это (и пока противоречий не нашли за 100 лет).

Это похоже на bootstrapping: нельзя проверить компилятор самим собой.
\end{thmbox}

\begin{center}
\textbf{Что неполнота означает и чего не означает}

\begin{tabular}{lp{6cm}}
\toprule
\textbf{Неполнота НЕ означает} & \textbf{Неполнота означает} \\
\midrule
<<Математика бесполезна>> & Есть принципиальные границы \\
<<Ничего нельзя доказать>> & Некоторые вопросы неразрешимы \\
<<Компьютеры не могут доказывать>> & Полная автоматизация невозможна \\
<<Нужно отказаться от формализма>> & Интуиция остаётся важной \\
\bottomrule
\end{tabular}
\end{center}

\textbf{Примеры неразрешимых утверждений:}
\begin{itemize}[nosep]
\item Континуум-гипотеза: $|\R| = \aleph_1$? — независима от ZFC
\item Аксиома выбора — независима от ZF
\item Некоторые комбинаторные утверждения (Пэрис-Харрингтон)
\end{itemize}

\textbf{Для инженера:} 99.9\% практических задач \textbf{не затронуты} неполнотой. Теоремы Гёделя важны для понимания \textbf{границ}, а не для ежедневной работы. Если алгоритм не находит решение — возможно, задача неразрешима, но обычно нужно просто искать лучше.

\end{document}
